\chapter{Conclusiones y trabajo futuro}

\section{Conclusión}
En este trabajo hemos buscado y estudiado los orígenes del plagio y los recursos disponibles para su detección motivados por resolver el problema de la copia entre documentos en R.
\newline
Dado que en nuestra búsqueda no encontramos ninguna forma verdaderamente efectiva de detectar el plagio en este lenguaje, nos decidimos a resolver el problema modificando una de las mejores herramientas de detección de plagio llamada JPLAG.
\newline
Para ello, como ya explicamos en los capítulos de Elección de Herramientas e Implementación, ha sido necesario crear un frontend, adaptar un analizador sintáctico, elegir los tokens más importantes de R y comunicar todo con el programa principal.
\newline
Hemos comparado entonces la nueva herramienta creada con las mejores opciones existentes para detectar la copia entre alumnos.
\newline
Los resultados obtenidos de esta comparación han sido satisfactorios ya que nuestra herramienta, al contrario que el resto, consigue identificar prácticamente todos los casos de plagio entre alumnos, cosa imposible hasta el momento a menos que el profesor o evaluador hiciese todas las comparaciones manualmente.
\newline
\newline
Se han cumplido por tanto todos los objetivos que especificamos en un principio en el capítulo de introducción (Sección 1.3).


\section{Trabajo Futuro}

Aunque el trabajo hecho permite que sea posible detectar la mayor parte de los plagios en R entre tareas de alumnos, hay diversos añadidos que se pueden hacer para mejorar la herramienta:

\begin{itemize}
\item Basándonos en los detectores de plagio en documentos de texto como Turnitin y Viper, se podría añadir una base de datos de documentos para poder comparar los archivos enviados con archivos similares encontrados por internet o con las mismas entregas de alumnos de años anteriores. Para hacer esto se tendría que lidiar con la inmensa carga extra de trabajo que le supondría al algoritmo de comparación. En caso de poder solucionar ese problema, esto sería una gran mejora ya que permitiría saber si el alumno se ha copiado de un documento externo diferente al del resto de sus compañeros.
\item Aunque JPLAG ya tenga soporte para los lenguajes más usados, estaría bien añadir soporte para más lenguajes.
\item Entrenar una red neuronal con casos que son y no son plagio para que JPLAG pueda decir con un alto porcentaje de certeza si los casos que hasta el momento teníamos que inspeccionar con mayor profundidad(las casos intermedios) son plagio o no. Esto ahorraría un gran volumen de trabajo al evaluador.
\end{itemize}













