\chapter{Planificación}

En este capítulo explico la metodología que se ha seguido para la realización de este trabajo, así como la organización de las tareas hecha siguiendo esta metodología.
\newline
Por último, concluimos este capítulo con el presupuesto del proyecto.


\section{Metodología usada.}

He usado una metodología ágil \cite{metodologia_agil} basándome en SCRUM \cite{scrum}: esta metodología consiste en mantener un contacto frecuente (cada 2 semanas aproximadamente) con nuestro cliente a través de reuniones para revisar el progreso conseguido en la anterior entrega y plantear los objetivos importantes para la próxima. Después de cada reunión se hace un sprint semanal o bisemanal para cumplir los objetivos planteados y se contacta en cualquier momento con el cliente si surge algún problema para replantear el sprint.
\newline
Esta metodología permite que haya un seguimiento del avance mucho más completo, acepta cambios durante cualquier etapa durante el desarrollo y soluciona los errores que surgen de forma más eficaz y con menos presión ya que no hay una entrega general a largo plazo.

En mi caso mi tutora de TFG es mi cliente y las reuniones se han llevado a cabo en su despacho y online (cuando no era posible hablar en persona) cada una o dos semanas dependiendo de la cantidad de trabajo asignada al sprint de la semana actual.


\section{Reparto de objetivos.}

Usando la metodología mencionada en la sección anterior estos han sido las tareas que se han llevado a cabo en cada sprint desde que comencé el trabajo el 4/4/18 hasta el 27/8/18:

\begin{itemize}
	\item \textbf{Primera reunión (4/4/18)}:
	\begin{itemize}
	\item Revisión bibliográfica sobre detección de plagio.
	\item Revisión de software de plagio en código fuente.
	\item Repaso de R.
	\item Búsqueda de software de plagio que funcione con R.
	\end{itemize}
	\item \textbf{Segunda reunión (18/4/18)}:
	\begin{itemize}
	\item Repaso rápido de uso de Latex.
	\item Empezar la redacción del proyecto en Latex.
	\item Ejecución y uso de la versión actual de JPLAG.
	\end{itemize}
	\item \textbf{Tercera reunión (25/4/18)}:
	\begin{itemize}
	\item Especificación de requisitos.
	\item Entender estructura de clases de JPLAG
	\end{itemize}
	\item \textbf{Cuarta reunión (9/5/18)}:
	\begin{itemize}
	\item Adaptar JPLAG para que detecte y use la extensión .r.
	\item Aprender cómo funciona en mayor profundidad Apache Maven para modificar correctamente el archivo pom que usa el proyecto en total.
	\end{itemize}
	\item \textbf{Quinta reunión (16/5/18)}:
	\begin{itemize}
	\item Aprender cómo consigue los tokens de los archivos JPLAG.
	\item Aprender ANTLR4.
	\item Empezar a crear una gramática que define a R para ANTLR4.
	\end{itemize}
	\item \textbf{Sexta reunión (1/6/18)}:
	\begin{itemize}
	\item Estudiar en mayor profundidad la estructura del lenguaje R con sus características específicas para poder crear correctamente su gramática.
	\item Redactar progreso hasta el momento.
	\end{itemize}
	\item \textbf{Séptima reunión (13/6/18)}:
	\begin{itemize}
	\item Crear los archivos .java dentro de nuestro frontend que comunican el parser con JPLAG.
	\item Implementar un parser inicial básico.
	\end{itemize}
	\item \textbf{Octava reunión (20/6/18)}:
	\begin{itemize}
	\item Conectar el parser creado con JPLAG de forma que reciba los tokens de los archivos que analiza.
	\item Elegir los tokens más relevantes del lenguaje (primer intento).
	\end{itemize}
	\item \textbf{Novena reunión (4/7/18)}:
	\begin{itemize}
	\item Realizar prueba inicial con versión actual del frontend.
	\item Redactar en la memoria los avances.
	\end{itemize}
	\item \textbf{Décima reunión (11/7/18)}:
	\begin{itemize}
	\item Modificar la gramática para poder seleccionar tokens más importantes.
	\item Elegir los tokens más relevantes del lenguaje (segundo intento).
	\item Probar de nuevo JPLAG con las modificaciones hechas.
	\end{itemize}
	\item \textbf{Undécima reunión (25/7/18)}:
	\begin{itemize}
	\item Última modificación a la gramática.
	\item Elegir los tokens más relevantes del lenguaje por última vez.
	\item Probar de nuevo JPLAG con las modificaciones hechas con los archivos de alumnos que obtenemos de la tutora.
	\item Anonimizar los archivos de alumnos.
	\end{itemize}
	\item \textbf{Duodécima reunión (6/8/18)}:
	\begin{itemize}
	\item Realizar los benchmarks con todas las entregas de los alumnos con nuestro frontend de R. 
	\item Realizar los benchmarks con todas las entregas de los alumnos con el frontend de texto plano.
	\item Realizar los benchmarks con todas las entregas de los alumnos con MOSS.
	\end{itemize}
	\item \textbf{Trigésima reunión (14/8/18)}:
	\begin{itemize}
	\item Escribir en la memoria la comparación de las pruebas realizadas.
	\item Terminar de redactar la memoria y corregir errores.
	\end{itemize}
	\item \textbf{Cuadragésima reunión (27/8/18)}:
	\begin{itemize}
	\item Últimas mejoras y corrección de errores en la memoria.
	\end{itemize}
\end{itemize}

\section{Presupuesto}

En este apartado planteamos un análisis del valor de los recursos usados para el desarrollo de este proyecto.
\subsection{Recursos materiales}
Aunque las herramientas que hemos estado manipulando para la detección de plagio son gratuitas, hemos hecho uso de una máquina local para realizar todas la pruebas, buscar información y redactar la memoria. Se trata de un ordenador de sobremesa con las siguiente especificaciones:
\begin{itemize}
\item Procesador: Intel i5 4670K (200 \euro)
\item Ram: 8 GB DDR3 (75 \euro)
\item Almacenamiento: HDD 1 TB 7200 rpm (50 \euro)
\item F. alimentación: LC-Power LC600H-12 (44 \euro)
\item Placa base: MSI H81M-P33 (60 \euro)
\item Caja: Aerocool V3XAD (30 \euro) 
\item Sistema Operativo: Windows 10 (proporcionado por la Universidad de Granada)
\end{itemize}
Esto asciende a un total de 459 \euro , aunque a esto hay que añadirle gastos como la electricidad, el material de papelería usado y la conexión a internet. 


\subsection{Recursos humanos}

Teniendo en cuenta que ha sido un trabajo de unos cinco meses de duración con 4 horas de media de trabajo diario 5 días a la semana, tenemos un total de 400 horas de trabajo.
\newline
El salario medio de un ingeniero informático con menos de un año de experiencia es de 17.248\euro  brutos al año.
\newline
400 horas de trabajo equivalen a trabajar durante aproximadamente tres meses por lo que la realización de este trabajo equivale a un total de 4.312\euro.


\subsection{Total}

Juntando los gastos estimados en recursos materiales (600 \euro) y en recursos humanos (4.312 \euro) obtenemos que el coste total aproximado de este proyecto ha sido de 4.912 \euro .


