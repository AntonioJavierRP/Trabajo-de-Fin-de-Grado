\chapter*{}
%\thispagestyle{empty}
%\cleardoublepage

\thispagestyle{empty}

\input{portada/portada_2}



\cleardoublepage
\thispagestyle{empty}

\begin{center}
{\large\bfseries Detección de Plagio en R}\\
\end{center}
\begin{center}
Antonio Javier Rodríguez Pérez\\
\end{center}

%\vspace{0.7cm}
\noindent{\textbf{Palabras clave}: detección de plagio, analizador léxico, analizador sintáctico, n-grama, algoritmo String Tiling}\\

\vspace{0.7cm}
\noindent{\textbf{Resumen}}\\

El plagio en código fuente y su detección es un problema que encontramos con gran frecuencia en la actualidad dado que hay una gran cantidad de software nuevo y programas en general que se crean cada día.
\newline
Este problema es incluso más preponderante en el ámbito educativo, ya que durante los últimos años se ha convertido en una práctica cada vez más habitual entre los estudiantes.
\newline
Es por esto que es importante que tengamos alguna forma de saber si estos programas son genuinos o tan solo una copia o una modificación de algo ya existente.
\newline
Para detectarlo existen diversas herramientas, pero no existe ninguna creada específicamente para el lenguaje R. 

\vspace{0.3cm}

En este trabajo se busca resolver este problema adaptando una de las herramientas disponibles llamada JPLAG para que funcione también con este lenguaje.
\newline
JPLAG permite determinar con mayor exactitud si existe plagio entre un conjunto de archivos de un mismo lenguaje que se le proporcione.
\newline
Modificar JPLAG para que sirva con R permitirá que la detección de plagio entre programas escritos en este lenguaje sea más rápida y eficiente que con cualquier otro método ya existente.

\cleardoublepage


\thispagestyle{empty}


\begin{center}
{\large\bfseries Plagiarism Detection in R}\\
\end{center}
\begin{center}
Antonio Javier Rodríguez Pérez\\
\end{center}

%\vspace{0.7cm}
\noindent{\textbf{Keywords}: Plagiarism Detection, Parser, Lexer, n-gram, String Tiling Algorithm}\\

\vspace{0.7cm}
\noindent{\textbf{Abstract}}\\

Plagiarism in source code and its detection is a problem we frecuently find nowadays due to the immense amount of new software and programs in general that are made everyday.
\newline
This problem is even more prevalent in the edutational field, since in recent years it has become an increasingly common practice among students.
\newline
For those reasons it's important to have a way of knowing whether that software is genuine or its just a copy or modification of something already existing.
\newline
In order to detect that plagiarism there are many tools avaliable, but there is not yet one created specifically for the language R.

\vspace{0.3cm}

In this proyect we try to solve this issue by adapting one of the avaliable tools called JPLAG so that it'll also function with R.
\newline
JPLAG allows us to determine with great precisión whether there is plagiarism among a set of files written in the same language, that you supply JPLAG with.
\newline
If JPLAG is modified so that it also detects R files, we will be able to detect plagiarism among programs in this language in a faster and more efficient way than any other existing method. 

\chapter*{}
\thispagestyle{empty}

\noindent\rule[-1ex]{\textwidth}{2pt}\\[4.5ex]

Yo, \textbf{Antonio Javier Rodríguez Pérez}, alumno de la titulación Grado en Ingeniería Informática de la \textbf{Escuela Técnica Superior
de Ingenierías Informática y de Telecomunicación de la Universidad de Granada}, con DNI 50643240T, autorizo la
ubicación de la siguiente copia de mi Trabajo Fin de Grado en la biblioteca del centro para que pueda ser
consultada por las personas que lo deseen.

\vspace{6cm}

\noindent Fdo: Antonio Javier Rodríguez Pérez

\vspace{2cm}

\begin{flushright}
Granada a 2 de Septiembre de 2018 .
\end{flushright}


\chapter*{}
\thispagestyle{empty}

\noindent\rule[-1ex]{\textwidth}{2pt}\\[4.5ex]

Dª. \textbf{Rocío Celeste Romero Zaliz}, Profesora del Departamento de Ciencias de la Computación e Inteligencia Artificial de la Universidad de Granada.


\vspace{0.5cm}

\textbf{Informa:}

\vspace{0.5cm}

Que el presente trabajo, titulado \textit{\textbf{DETECCIÓN DE PLAGIO EN R}},
ha sido realizado bajo su supervisión por \textbf{Antonio Javier Rodríguez Pérez}, y autorizo la defensa de dicho trabajo ante el tribunal
que corresponda.

\vspace{0.5cm}

Y para que conste, expide y firma el presente informe en Granada a 2 de Septiembre de 2018 .

\vspace{1cm}

\textbf{La directora:}

\vspace{5cm}

\noindent \textbf{Rocío Celeste Romero Zaliz}

\chapter*{Agradecimientos}
\thispagestyle{empty}

       \vspace{1cm}

Gracias a mi madre y a mi padre, Josefa y Antonio, por insistir en mi educación y ayudarme siempre que lo necesitase.

\bigskip

A mi hermana Laura por sus recomendaciones y por estar siempre ahí.

\bigskip

A Rocío, mi tutora, por guiarme y aconsejarme durante la creación de este proyecto.

\bigskip

A mis amigos de la facultad, con los que he pasado cuatro duros años de estudio. De estos amigos debo mencionar a Javi, Sergio y Jesús ya que son con los que más tiempo he pasado y son los que definen en gran parte quien soy.

\bigskip

Y por último, a Alba, con ella he pasado la mayor parte del tiempo durante estos años de carrera y ha conseguido que vea el mundo de forma diferente.



\chapter*{Declaración de autoría y originalidad del TFG}
\thispagestyle{empty}

Yo, Antonio Javier Rodríguez Pérez, con DNI 50643240T, declaro que el presente documento ha sido realizado por mí y se basa en mi propio trabajo, a menos que se indique lo contrario. No se ha utilizado el trabajo de ninguna otra persona sin el debido reconocimiento. Todas las referencias han sido citadas y todas las fuentes de información y conjuntos de datos han sido específicamente reconocidos.

\vspace{6cm}

\noindent Fdo: Antonio Javier Rodríguez Pérez

\vspace{2cm}

\begin{flushright}
Granada a 2 de Septiembre de 2018 .
\end{flushright}